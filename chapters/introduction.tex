% =====================================================
% INTRODUCTION GENERALE
% =====================================================
\chapter*{Introduction}
\addcontentsline{toc}{chapter}{Introduction}
\textbf{Objectif.} Valider la robustesse d'une application microservices face à des défaillances
contrôlées, conformément au sujet \#6 « Résilience et Chaos Engineering » (déployer, injecter des
fautes, mesurer l'impact, expérimenter la tolérance, conclure).

\textbf{Approche.} (i) Déploiement Docker Compose avec réplicas et load balancer ; (ii) Observabilité
Micrometer/Prometheus/Grafana/cAdvisor ; (iii) Campagnes de chaos via Pumba ; (iv) Mesures
de performance (JMeter) et métriques métier (ex.\ nombre de commandes/min) ; (v) Stratégies de
tolérance (load balancing, réplication, circuit breakers) et analyse.

\textbf{Contributions.} Un environnement reproductible incluant configuration des services, des
dashboards, et un protocole expérimental avec banque de scénarios.

\paragraph{Mini-conclusion.}
Le cadre expérimental est défini : nous allons d'abord déployer l'application, puis
introduire des pannes mesurées, avant de proposer et d'évaluer des mécanismes de tolérance.