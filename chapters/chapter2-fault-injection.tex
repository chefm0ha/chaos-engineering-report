% =====================================================
% CHAPITRE 2 : I N J E C T I O N  D E  F A U T E S
% =====================================================
\chapter{Injection de fautes avec Pumba (focus expérimental)}
\label{chap:pumba}

\section{Contexte et hypothèses}
Nous exécutons les tests sous \textbf{Docker Compose} avec \textbf{Spring Cloud Gateway} comme unique point d'entrée (pas de HAProxy/Nginx externe). 
Chaque microservice possède \textbf{une seule instance} (pas de \texttt{deploy.replicas}) et aucune politique de redémarrage (\texttt{restart} non définie). 
En conséquence, une panne du conteneur \texttt{product-service} provoque une indisponibilité totale de ce service (SPOF).

\section{Cibles et outillage}
\begin{itemize}
  \item \textbf{Cible} : \texttt{product-service} (regex exacte \texttt{re2:\^{}product-service\$}).
  \item \textbf{Réseau} : \texttt{ecommerce-network} (bridge).
  \item \textbf{Outils} : Pumba \texttt{gaiaadm/pumba:0.9.0} ; image \texttt{gaiadocker/iproute2} pour \texttt{tc}.
\end{itemize}

\section{Scénarios de pannes}
\subsection{Arrêt temporaire (Stop)}
\begin{lstlisting}[caption={Arrêt 30s (graceful downtime)}]
docker run -it --rm \
  -v /var/run/docker.sock:/var/run/docker.sock \
  --network ecommerce-network \
  gaiaadm/pumba:0.9.0 \
  stop --duration 30s --time 10s re2:^product-service$
\end{lstlisting}
Effet attendu : indisponibilité de \texttt{product-service} pendant 30s, puis reprise immédiate.

\subsection{Crash (Kill)}
\begin{lstlisting}[caption={Kill gracieux vs brutal}]
# SIGTERM (gracieux)
docker run -it --rm -v /var/run/docker.sock:/var/run/docker.sock \
  gaiaadm/pumba:0.9.0 \
  kill --signal SIGTERM re2:^product-service$

# SIGKILL (brutal)
docker run -it --rm -v /var/run/docker.sock:/var/run/docker.sock \
  gaiaadm/pumba:0.9.0 \
  kill --signal SIGKILL re2:^product-service$
\end{lstlisting}
Sans politique \texttt{restart}, le conteneur reste \textbf{DOWN} après \texttt{kill}.

\subsection{Latence réseau (Netem Delay)}
\begin{lstlisting}[caption={Latence 500ms ±100ms, 2 minutes}]
docker run -it --rm \
  -v /var/run/docker.sock:/var/run/docker.sock \
  --network ecommerce-network \
  gaiaadm/pumba:0.9.0 netem \
    --tc-image gaiadocker/iproute2 \
    --duration 2m delay \
      --time 500 --jitter 100 --distribution normal \
    re2:^product-service$
\end{lstlisting}

\subsection{Perte de paquets (Netem Loss)}
\begin{lstlisting}[caption={Perte 20\% de paquets, 2 minutes}]
docker run -it --rm \
  -v /var/run/docker.sock:/var/run/docker.sock \
  --network ecommerce-network \
  gaiaadm/pumba:0.9.0 netem \
    --tc-image gaiadocker/iproute2 \
    --duration 2m loss --percent 20 \
    re2:^product-service$
\end{lstlisting}

\subsection{Chaos continu (micro-coupures)}
\begin{lstlisting}[caption={Cycles de micro-latence (15s toutes 30s)}]
docker run -it --rm \
  -v /var/run/docker.sock:/var/run/docker.sock \
  gaiaadm/pumba:0.9.0 \
  --interval 30s --random netem \
    --tc-image gaiadocker/iproute2 \
    --duration 15s delay --time 200 \
    re2:^product-service$
\end{lstlisting}

\paragraph{Plan d'exécution JMeter.}
Baseline (5 min) → Injection (2--5 min) → Recovery (5 min), avec:
\begin{itemize}
  \item Test \textbf{Product}: 100 utilisateurs, ramp-up 60s, 20 itérations.
  \item Test \textbf{E2E}: 20 utilisateurs, ramp-up 60s, 5 itérations.
\end{itemize}

\begin{figure}[H]
\centering
\placeholder{0.85\textwidth}{5cm}
\caption{Chronologie Baseline/Chaos/Recovery avec marqueurs d'événements.}
\end{figure}